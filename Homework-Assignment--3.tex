% Options for packages loaded elsewhere
\PassOptionsToPackage{unicode}{hyperref}
\PassOptionsToPackage{hyphens}{url}
%
\documentclass[
]{article}
\title{Homework Assignment \#3 - Distributional Consequences of Climate
Policy}
\author{Daija Odom}
\date{5/2/2022}

\usepackage{amsmath,amssymb}
\usepackage{lmodern}
\usepackage{iftex}
\ifPDFTeX
  \usepackage[T1]{fontenc}
  \usepackage[utf8]{inputenc}
  \usepackage{textcomp} % provide euro and other symbols
\else % if luatex or xetex
  \usepackage{unicode-math}
  \defaultfontfeatures{Scale=MatchLowercase}
  \defaultfontfeatures[\rmfamily]{Ligatures=TeX,Scale=1}
\fi
% Use upquote if available, for straight quotes in verbatim environments
\IfFileExists{upquote.sty}{\usepackage{upquote}}{}
\IfFileExists{microtype.sty}{% use microtype if available
  \usepackage[]{microtype}
  \UseMicrotypeSet[protrusion]{basicmath} % disable protrusion for tt fonts
}{}
\makeatletter
\@ifundefined{KOMAClassName}{% if non-KOMA class
  \IfFileExists{parskip.sty}{%
    \usepackage{parskip}
  }{% else
    \setlength{\parindent}{0pt}
    \setlength{\parskip}{6pt plus 2pt minus 1pt}}
}{% if KOMA class
  \KOMAoptions{parskip=half}}
\makeatother
\usepackage{xcolor}
\IfFileExists{xurl.sty}{\usepackage{xurl}}{} % add URL line breaks if available
\IfFileExists{bookmark.sty}{\usepackage{bookmark}}{\usepackage{hyperref}}
\hypersetup{
  pdftitle={Homework Assignment \#3 - Distributional Consequences of Climate Policy},
  pdfauthor={Daija Odom},
  hidelinks,
  pdfcreator={LaTeX via pandoc}}
\urlstyle{same} % disable monospaced font for URLs
\usepackage[margin=1in]{geometry}
\usepackage{graphicx}
\makeatletter
\def\maxwidth{\ifdim\Gin@nat@width>\linewidth\linewidth\else\Gin@nat@width\fi}
\def\maxheight{\ifdim\Gin@nat@height>\textheight\textheight\else\Gin@nat@height\fi}
\makeatother
% Scale images if necessary, so that they will not overflow the page
% margins by default, and it is still possible to overwrite the defaults
% using explicit options in \includegraphics[width, height, ...]{}
\setkeys{Gin}{width=\maxwidth,height=\maxheight,keepaspectratio}
% Set default figure placement to htbp
\makeatletter
\def\fps@figure{htbp}
\makeatother
\setlength{\emergencystretch}{3em} % prevent overfull lines
\providecommand{\tightlist}{%
  \setlength{\itemsep}{0pt}\setlength{\parskip}{0pt}}
\setcounter{secnumdepth}{-\maxdimen} % remove section numbering
\usepackage{booktabs}
\usepackage{longtable}
\usepackage{array}
\usepackage{multirow}
\usepackage{wrapfig}
\usepackage{float}
\usepackage{colortbl}
\usepackage{pdflscape}
\usepackage{tabu}
\usepackage{threeparttable}
\usepackage{threeparttablex}
\usepackage[normalem]{ulem}
\usepackage{makecell}
\usepackage{xcolor}
\ifLuaTeX
  \usepackage{selnolig}  % disable illegal ligatures
\fi

\begin{document}
\maketitle

\textbf{1. One kWh of electricity emits 0.85 pounds of CO\_2\_. Assuming
that the interim SCC correctly reflects the total social cost of one
metric ton of CO\_2\_, what is the marginal externality cost per kWh of
electricity?}

MEC = \$ 1.97/kWh

\textbf{2. What is the aggregate monthly demand curve for electricity?
What is the supply curve for electricity? What is the ``benefit'' to
consumers under the status quo? What is the ``benefit'' to producers
under the status quo? What is the environmental cost under the status
quo?}

By computing the aggregate demand demand for both low and high income
consumers, I solved for the free market equilibrium quantity of
electricity demanded \ensuremath{5.3671947\times 10^{5}}kWh for an
electricity price of 10 cents.

\textbf{3. How is the current consumer benefit divided between ``high''
and ``low'' income consumers?}

\begin{verbatim}
## Warning: Removed 4 row(s) containing missing values (geom_path).
\end{verbatim}

\begin{verbatim}
## Warning: Removed 26 row(s) containing missing values (geom_path).
\end{verbatim}

\includegraphics{Homework-Assignment--3_files/figure-latex/unnamed-chunk-9-1.pdf}
\textbf{4. Derive the optimal tax in cents per kWh using the interim
SCC. Noting that recent research has shown the poor face a
disproportionate share of the impacts from climate change, assume that
the climate externality is borne entirely by the ``low'' income group.
What would be the effects of this tax on: (a) The amount of electricity
produced and consumed, (b) The price of electricity, (c) Overall welfare
of the ``high'' income consumers, (d) Overall welfare of ``low'' income
consumers, (e) Power suppliers (i.e., electricity producers), (f) Total
environmental damage, (g) Total tax revenue generated? }

\textbf{5. Now, assume that all revenue from the electricity tax will be
redistributed to the consumers in proportion to their pre-tax
consumption. For example, if 80\% of the electricity was consumer by
``high'' income consumers, then they get 80\% of the tax revenue.
Additionally, consider the fact that current scientific evidence
suggests the true SCC may be much higher than \$51. For a range of SCC
values (\$51, \$75, \$100, \$125, and \$150 per metric ton of CO\_2\_),
calculate the effects of an SCC-based electricity tax on: (a) Overall
welfare of ``high'' income consumers, (b) Overall welfare of ``low''
income consumers, (c) Electricity producers.}

\textbf{6.Suppose the ``high'' income group has access to expensive home
solar generation. This lowers the electricity demand curve for the
``high'' income group by half (vertically). Under this new demand: (a)
What is the total electricity consumption, (b) What is the total
environmental externality, (c) What market value of the electricity tax
makes the total environmental damage the same as the damage when solar
panels are available to the high income group?}

\end{document}
